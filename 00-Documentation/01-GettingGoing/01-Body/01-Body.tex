\section*{\color{RedViolet} Introduction}
This document summarises the steps needed to set-up and run the LhARA
costing tool (LCT).
Instructions for the set up of a user copy of the application are
presented first.
The user set up instructions assume that the python software has been
installed elsewhere on the cluster and that python 3 has been
installed.

\vskip 0.5cm
\noindent{\color{DarkYellow} \rule[0mm]{\textwidth}{0.43pt}}

\section*{\color{RedViolet} Setting up a user application}
The first step is to create a directory from which the user wishes to
run the costing tool; in the following the path to the user's
directory will be labelled {\tt UserDir}.
The following instructions are to be executed from within this
directory.

To create a user application requires that the LCT has been
installed.
Assume that the path to the LCT top directory on the users system is:
\begin{center}
  {\tt <LhARA\_path>}
\end{center}
Then, to install the files needed to run the tool on sample test data
the user needs to execute the command:
\begin{center}
  {\tt <LhARA\_path>/05-Export/export-LhARA-costing-tool.bash}
\end{center}
This script with copy the directories and costing tool script into the
user's current directory.
As a concrete example, the LCT has been installed at:
\begin{center}
  {\tt /home/hep/longkr/CCAP/LhARA/01-Costing-tool/LhARA-costing-tool}
\end{center}
on the Imperial HEP Group's linux cluster.
The command to install the user application on this cluster is
therefore:
\begin{center}
  {\tt\tiny /home/hep/longkr/CCAP/LhARA/01-Costing-tool/LhARA-costing-tool/05-Export/export-LhARA-costing-tool.bash}
\end{center}

\subsection*{\color{DarkGreen} What is installed}
The following directories and files are installed when the ``export''
script is executed:
\begin{description}
  \item {\tt Control/} \\
    The {\tt Control/} directory contains one file:
    {\tt LhARA-costing-tool-control.csv}.
    This ``comma separated values'' (csv) file contains parameters
    that control the execution of the code.
    The file can be edited using EXCEL and must be saved back into the
    {\tt Control/} directory in csv format.
  \item {\tt Reports/} \\
    On installation this directory will be empty.
    When the costing tool is run, reports will be placed in this
    directory.
    The tool will create a director with the name corresponding to the
    date the tool is run in the format ``{\tt yyyy-mm-dd}''.
  \item {\tt StaffDatabase/} \\
    This directory will contain one file: {\tt StaffDatabase.csv}.
    At input a dummy list of staff will be contained in the csv file.
    The file must be updated in consultation with the LhARA project
    team.
    The information contained in the file is confidential and only the
    file containing dummy entries will be circulated with the costing
    tool distribution.
  \item {\tt WorkPackages} \\
    The {\tt WorkPackages} directory contains the csv files that
    specify the tasks, staff, and equipment required to deliver the
    programme defined for the particular work package.
    The csv files will be derived from EXCEL files maintained by the
    work package managers.
  \item {\tt run-LhARA-costing-tool.bash} \\
    Executing this script will perform the costing of the work
    packages contained in the {\tt WorkPackages} directory, using the
    staff information contained in the {\tt StaffDatabase/} directory
    and with the control flags defined in the {\tt Control}
    directory.
    The script is run with the command:
    \begin{center}
      {\tt ./run-LhARA-costing-tool.bash}
    \end{center}
    The command must be run from {\tt UserDir}.
\end{description}

\vskip 0.5cm
\noindent{\color{DarkYellow} \rule[0mm]{\textwidth}{0.43pt}}

\section*{\color{RedViolet} For those who wish to install the code}

\subsection*{\color{DarkGreen} Getting the code}
LCT is maintained using the GitHub version-control system.
The latest release can be downloaded from:
\begin{center}
  {\tt https://github.com/longkr/LhARA-costing-tool.git}
\end{center}

\subsection*{\color{DarkGreen} Dependencies and required packages}
LCT requires the following packages:
\begin{itemize}
  \item Python modules: \verb+pandas+.
\end{itemize}
It may be convenient to run LCT in a ``virtual environment''.
To set this up, after updating your python installation to python~3,
and installing root, execute the following commands:
\begin{enumerate}
  \item \verb+python3 -m venv --system-site-packages venv+
    \begin{itemize}
      \item This creates the director \verb+venv+ that contains files
        related to the virtual environment.
    \end{itemize}
  \item \verb+source venv/bin/activate+
  \item \verb+python -m pip install pandas+
\end{enumerate}
To exit from the virtual environment, execute the command
\verb+deactivate+. \\
\noindent
The command \verb+source venv/bin/activate+ places you back
into the virtual environment.

\subsection*{\color{DarkGreen} Unpacking the code, directories, and running the tests}
After downloading the package from GitHub, or cloning the repository,
you will find a ``\verb+README.md+'' file which provides some orientation
and instructions to run the code.
In particular, a \verb+bash+ script ``\verb+startup.bash+'' is
provided which:
\begin{itemize}
  \item Sets the ``\verb+LhARAPATH+'' environment variable so that the
    files that hold constants etc. required by the code can be
    located; and
  \item Adds ``\verb+01-Code+'' (see below) to the PYTHONPATH.
    The scripts in "02-Tests" (see below) may then be run with the
    command "python 02-Tests/\textless\,filename\,\textgreater.py".
\end{itemize}
Below the top directory, the directory structure in which the code is
presented is:
\begin{description}
  \item\verb+01-Code+: contains the python implementation as a
    series of modules.
    Each module contains a single class or a related set of methods.
  \item\verb+02-Tests+: contains self-contained test scripts that
    run the various methods and simulation packages defined in the
    code directory.
\end{description}
The instruction in the \verb+README.md+ file should be followed to set
up and run the code.
  
\subsection*{\color{DarkGreen} Making a contribution}
LCT is archived in the git repository \verb+longkr/LhARA-costing-tool+.
To clone the code using
\verb+git clone+ you will need your own account on GitHub and
permission to clone the code. 
